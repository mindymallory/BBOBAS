\documentclass[]{elsarticle} %review=doublespace preprint=single 5p=2 column
%%% Begin My package additions %%%%%%%%%%%%%%%%%%%
\usepackage[hyphens]{url}
\usepackage{lineno} % add
\providecommand{\tightlist}{%
  \setlength{\itemsep}{0pt}\setlength{\parskip}{0pt}}

\bibliographystyle{elsarticle-harv}
\biboptions{sort&compress} % For natbib
\usepackage{graphicx}
\usepackage{booktabs} % book-quality tables
%% Redefines the elsarticle footer
%\makeatletter
%\def\ps@pprintTitle{%
% \let\@oddhead\@empty
% \let\@evenhead\@empty
% \def\@oddfoot{\it \hfill\today}%
% \let\@evenfoot\@oddfoot}
%\makeatother

% A modified page layout
\textwidth 6.75in
\oddsidemargin -0.15in
\evensidemargin -0.15in
\textheight 9in
\topmargin -0.5in
%%%%%%%%%%%%%%%% end my additions to header

\usepackage[T1]{fontenc}
\usepackage{lmodern}
\usepackage{amssymb,amsmath}
\usepackage{ifxetex,ifluatex}
\usepackage{fixltx2e} % provides \textsubscript
% use upquote if available, for straight quotes in verbatim environments
\IfFileExists{upquote.sty}{\usepackage{upquote}}{}
\ifnum 0\ifxetex 1\fi\ifluatex 1\fi=0 % if pdftex
  \usepackage[utf8]{inputenc}
\else % if luatex or xelatex
  \usepackage{fontspec}
  \ifxetex
    \usepackage{xltxtra,xunicode}
  \fi
  \defaultfontfeatures{Mapping=tex-text,Scale=MatchLowercase}
  \newcommand{\euro}{€}
\fi
% use microtype if available
\IfFileExists{microtype.sty}{\usepackage{microtype}}{}
\ifxetex
  \usepackage[setpagesize=false, % page size defined by xetex
              unicode=false, % unicode breaks when used with xetex
              xetex]{hyperref}
\else
  \usepackage[unicode=true]{hyperref}
\fi
\hypersetup{breaklinks=true,
            bookmarks=true,
            pdfauthor={},
            pdftitle={What Nearby and Deferred Quotes Tell Us about Linkages and Adjustments to Information},
            colorlinks=true,
            urlcolor=blue,
            linkcolor=magenta,
            pdfborder={0 0 0}}
\urlstyle{same}  % don't use monospace font for urls
\setlength{\parindent}{0pt}
\setlength{\parskip}{6pt plus 2pt minus 1pt}
\setlength{\emergencystretch}{3em}  % prevent overfull lines
\setcounter{secnumdepth}{0}
% Pandoc toggle for numbering sections (defaults to be off)
\setcounter{secnumdepth}{0}
% Pandoc header


\usepackage[nomarkers]{endfloat}

\begin{document}
\begin{frontmatter}

  \title{What Nearby and Deferred Quotes Tell Us about Linkages and Adjustments
to Information}
    \author[University of Illinois]{Mindy L. Mallory\corref{c1}}
   \ead{mallorym@illinois.edu} 
   \cortext[c1]{Corresponding Author}
    \author[University of Illinois]{Philip Garcia}
   \ead{p-garcia@illinois.edu} 
  
    \author[University of Illinois]{Teresa Serra}
   \ead{tserra@illinois.edu} 
  
      \address[University of Illinois]{Agricultural and Consumer Economics, 1301 W Gregory Dr, Urbana, IL,
61801}
  
  \begin{abstract}
  The recent `Financialization' of commodity futures markets, increases in
  biofuel production, and climate change potentially have imposed profound
  shifts in the way commodity futures markets operate. This article
  examines the corn market quote-by-quote to develop metrics on liquidity
  and transmission of information. The metrics are based on insights
  derived from sequential trading models on single securities, index
  futures on a basket of securities, and special features of commodity
  futures markets. Correlation between quote revisions in nearby and
  deferred contracts measure information-based activity, and correlations
  between revisions of the time lagged nearby and deferred maturity
  measure the speed at which information is transmitted among the
  different futures maturities. Information-based trading results in near
  perfect correlation between revisions to bids and offers in nearby and
  deferred contracts. Within one second, information is fully transmitted
  from nearby to deferred contracts.
  \end{abstract}
   \begin{keyword} market, microstructure \sep \end{keyword}
 \end{frontmatter}

\section{Introduction}\label{introduction}

There has been recent concern about whether and how the
`Financialization of Commodity Markets' has impacted market efficiency
and efficacy in the traditional roles of risk mitigation, coordinating
production, and coordinating consumption through time (Irwin and Sanders
2011; Cheng and Xiong 2013; Irwin and Sanders 2012; Henderson, Pearson,
and Wang 2015). Further, the recent increase in the production of
biofuel from food commodities and volatile crude oil prices has changed
the relationship between food and energy commodities (Serra and
Zilberman 2013; M. L. Mallory, Irwin, and Hayes 2012; Gardebroek and
Hernandez 2013; Vacha et al. 2013; Avalos 2014; Trujillo-Barrera et al.
2012). Additionally, climate change, rising demand for agricultural
commodities, and volatile inventories and exchange rates have imposed
structural changes in commodity markets (Balcombe, Prakash, and others
2011; Gilbert and Morgan 2010; A. Prakash, Gilbert, and others 2011).

These issues represent potentially profound shifts in the way commodity
markets operate, and the articles cited above have considered their
implications. However, how these changes affect commodity markets
tick-by-tick and quote-by-quote needs to be considered. Since global
price discovery occurs on global futures exchanges for the major food
commodities, a deep consideration of these changes on trading activity,
patterns, and consequences is warranted. We use ``high frequency data''
(time stamped to the second), in order to capture faster price change
adjustments taking place after significant technical developments in
trading platforms in the second half of the 2000s, characterized by high
speed trading.

Price analysis can be classified into structural and non-structural
studies. While structural models rely on economic theory, non-structural
analyses identify empirical regularities in the data. The approach
throughout this article is non-structural. We employ this approach
primarily because there is scant market microstructure literature
developed with the particular characteristics of commodity futures
markets in mind. In this article, we are motivated to develop initial
metrics of information-based activity in commodity markets. We
anticipate this work will lead to future developments in the
microstructure of commodity markets literature.

Even how to develop simple metrics of information-based activity from
standard microstructure models is not obvious because standard models of
trading securities are not necessarily directly applicable to commodity
futures markets. For example, in commodities futures markets several
contracts with different maturities trade in the marketplace, each
reacting to information- and liquidity-motivated trades. Each contract
responds to information-based shocks because there is a cost to store
the physical commodity through time (Working 1948; Working 1949; Brennan
1958). Further, each contract maturity attracts different levels of
liquidity, and it is not known what impact a lack of liquidity has on
information transmission up the forward curve.

The metrics we develop in this article on liquidity and transmission of
information are based on insights we combined from the sequential
trading models on single securities, index futures based on a basket of
securities, and some of the features of commodity futures markets
described in the preceding paragraph. Using the standard sequential
trading result that quote revisions only occur if liquidity providers
have updated their beliefs about the value of the security after
observing order flows, the correlation between quote revisions in nearby
and deferred contracts can be used to measure information-based
activity, and correlations between revisions of the time lagged nearby
and deferred maturity can be used to measure the speed at which
information is transmitted among the different futures maturities. This
metric is sensible in commodity futures markets but not in a market for
a single security, because futures markets have multiple maturity
contracts that should respond to information in a very similar and
predictable way. We find information is fully transmitted along the
forward curve so that nearby and distant contracts have fully adjusted
to new information within one second.

The remainder of the article is organized as follows. First, we provide
a background of the sequential trading and index futures microstructure
literature and describe the conceptual framework that motivates our
interpretation of correlations of quote revisions as a metric of
information-based activity. Next we describe the data and report the
results of our analysis. Finally, we offer concluding remarks.

\section{Literature Review}\label{literature-review}

The literature on how information affects liquidity in securities
markets is long and rich.\footnote{The interested reader can refer to
  O'Hara (1995) for an excellent and detailed overview of the evolution
  of this literature.} Bagehot (1971) is regarded as the first to
demonstrate that a bid-ask spread (BAS) arises when asymmetric
information is present even if inventory and transactions costs are
assumed to be zero. Copeland and Galai (1983) buid upon Bagehot's work
by assuming that a specific proportion of traders are informed. Knowing
this, the market maker adjusts his quoted bids and offers to maximize
expected profit. Copeland and Galai's model, however, does not account
for the fact that the trades themselves can reveal information about
whether or not traders are informed. Glosten and Milgrom (1985)
formalize this concept and develop a model where the market maker
adjusts his beliefs based on the trades that occur. The market maker
knows that at least some of the traders are informed so sell orders
revise the market maker's belief downward about the value of the
security and buy orders revise his belief upward. They show that the
spread is increasing in the proportion of informed traders, and there is
a point at which too many informed traders require the market maker to
set the spread so wide, that trade does not occur and the market halts
(an example of the famous ``Market for Lemons'' described by Akerlof
(1970)).

Easley and O'Hara (1987) and Easley and O'Hara (1992) incorporate trade
size and its effect to a model similar to Glosten and Milgrom. A market
maker must set breakeven bid and offer quotes knowing that he faces a
certain proportion of informed traders who only trade if they receive a
signal that an information event has occurred, and a certain proportion
of uninformed traders who do not receive an information signal but
occasionally need to trade for liquidity reasons. Both informed and
uninformed traders can choose between a large and small block trading
size. This model setup leads to two types of equilibria: a separating
equilibrium where informed traders only trade in large quantities and a
pooling equilibria where informed traders may trade both large and small
quantities. This model setup of information uncertainty and asymmetric
information leads to the market maker updating his beliefs about the
value of the security (and therefore his quotes) based on the order flow
he observes in the market. For example, in a separating equilibrium a
large trading block causes the market maker to revise upward his
expectation that an information event has occurred (since informed
traders do not transact at small sizes). This contrasts with the pooling
equilibrium where informed traders place small orders to prevent the
market maker from updating his beliefs that an information event has
occurred.

Hasbrouck (2006) provides an overview of how Easley, Hvidkjaer, and
O'Hara (2002) and Easley, Kiefer, and O'Hara (1997) use the Easley and
O'Hara models of informed trading to develop a measure of the
probability of informed trading (PIN). This measure, though, is
estimated solely based on the sequence of order arrivals, where a trade
is labeled as buyer initiated if the trade occurs above the midpoint of
the quoted spread and seller initiated if the trade occurs below the
midpoint of the quoted spread. Numerous studies have documented that
there may be problems with downward bias in the estimated PIN (Yan and
Zhang 2012; Vega 2006; Boehmer, Grammig, and Theissen 2007) and
estimating information-based trading in this way ignores some aspects of
futures markets discussed above that are not present in securities
markets. For these reasons we seek an alternative to the PIN measure of
information-based trading in commodity futures.

To our knowledge, there are no market microstructure models that
explicitly take into account the features of commodity futures markets.
The closest models come from work on index futures that cover a basket
of securities. Most prominent is the work by Kumar and Seppi (1994) who
assume \emph{N} different non-dividend paying securities and an index
futures contract on a buy-and-hold portfolio of a subset of these
stocks. In Kumar and Seppi's model, specialists in the cash market
observe a signal, and floor traders of the futures index observe a
signal about the value of the index but not the individual securities. A
key feature they build into the model is a lag in the information
transmittal between the cash and futures markets because specialists
only observe order flows from their own market, and not the other. This
lag in information transmittal allows for arbitrageurs, who possess
faster telecommunication technologies, to learn from transactions in
both markets and make profitable trades in the cash and futures markets.
These arbitrageurs are analogous to spread traders who trade in both
nearby and deferred contracts hoping to profit on relative price
movements.

There are some important distinctions between the arbitrageurs as
proposed in the Kumar and Seppi model and spread traders in a futures
market. Namely, the basis between a composite of cash security prices
and the price of a futures index of the same basket should behave in
very predictable ways (the basis, in theory, should only vary with
interest rates and expected changes in dividend yields if information is
symmetric). In contrast, the spread between the prices of two commodity
futures contracts with different maturities depends on many more
uncertain structural variables: e.g., domestic and international
consumption, exchange rates, production or distribution bottlenecks, or
weather. The arbitrageurs in Kumar and Seppi's model need only to wait
for others in the marketplace to learn to profit. The futures market
spread trader entertains much more risk in betting on relative price
changes between two futures maturities.

In the next section we draw insights from the sequential trading models
described above to generate empirical predictions about the correlations
between revisions to bids and offers of nearby and deferred maturity
commodity futures contracts.

\section{Conceptual Framework}\label{conceptual-framework}

In this section we develop a conceptual framework for how the role of
liquidity-based activity versus information-based activity should affect
quote revisions in a commodity futures market. Using insights from the
Easley and O'Hara models, along with features of commodity futures
markets, we generate empirical predictions about the correlations
between revisions to quotes in the nearby and deferred maturity
commodity futures contracts.

First, consider an absence of information. In the Easley and O'Hara
sequential trader models, the market maker revises his quotes only when
he updates his belief that the value of the security has changed.
Therefore, we interpret no changes in revisions to bids (offers) as
indicative of no information having arrived to the market. Any
transactions that occur at these prices, the market maker believes are
conducted by uninformed traders demanding liquidity.

Conversely, when we observe revisions to the bid or offer, we can infer
that the market maker from the Easley and O'Hara models has updated
beliefs about information arrival to the market based on past order
flows. These revisions to the bid and offer we interpret as indicative
of information arriving to the market.

Now we discuss features of futures markets that we can utilize when
considering revisions to nearby and deferred contract quotes. First, in
actively traded commodity futures markets there is no market maker, but
there are entities who actively supply liquidity to the market under a
variety of motives. Since the Easley and O'Hara models consider a
competitive market maker, it is irrelevant whether there is one market
maker in the traditional sense or a large number of traders providing
liquidity. Second, when market makers revise their beliefs that an
information event has arrived to the market, they know it affects
futures contracts of all maturities so quotes must be revised in all
contracts.

This should induce a high degree of correlation between bid and offer
revisions when an information event arrives. Further, one market maker
would revise bids and offers on futures contracts of all maturities at
the same time they update beliefs about an information event having
occurred. As a practical matter, many independent traders provide
liquidity to the market at any given time, so it is not clear that the
Bayesian updating described in the Easley and O'Hara models will happen
in all maturities simultaneously. Therefore, it is of interest to
consider the relationship between revisions to quotes in the nearby
contract at different time lags) and revisions to quotes in deferred
maturity contracts.

\section{Data}\label{data}

The data used in this analysis come from the CME Group's Top of Book
(BBO) database for Globex corn futures quotes and transactions from
01/14/2008-11/4/2011. The data contain the best bid, bid size, best
offer, offer size, last trade price, and last trade size of the order
book for each active futures contract, time-stamped to the second.

Table 1 shows the first ten entries to our data after manipulating the
raw BBO data set from CME Group to display the entire top of the book on
one line with the appropriate time stamp. The first column is the
time-stamp, the second column is the trade sequence number, which the
CME Group gives to individual trades to identify separate orders that
arrive on the same second. The third column, SYMBOL, identifies which
futures maturity the observation represents. In this case, 1003 stands
for March 2010, with the first two characters representing the year and
the second two characters representing the month. The fourth column,
OFRSIZ, is the number of contracts quoted at the best offer price. The
fifth column, OFR, is the best offered price. The sixth column, BIDSIZ,
is the number of contracts quoted at the best bid price; the last
column, BID, is the best bid price. For each date in our sample, we
consider the first to mature (nearby), one, two, and three contracts
deferred. We defined the nearby contract to be the next contract to
expire unless the date was after the 20th of the month prior to
expiration. Then we rolled the nearby to the next to expire contract. We
rolled the series on the 20th to avoid decreasing volume as the contract
neared the delivery period. We also excluded the September futures
contract from our analysis because of low trading volume.\footnote{September
  experiences low trading volumes because deliveries on this contract
  sometimes (but not usually) can come from early new crop harvest,
  making its price relative to the traditional new crop contract,
  December, hard to predict.}

Figure 1 displays average transaction price per day, number of revisions
to the ask, number of revisions to the bid, and number of transactions
--- all in the nearby contract. The first panel demonstrates that the
time period examined was characterized by volatility, uncertainty, and
rapid increases in prices in the beginning and end of the sample. Note
that prices increased to a peak of nearly \$8.00 per bushel in 2008, a
time a time that saw a broad class of commodity markets exhibiting
similar rapid price increases. Then a relatively stable period from 2009
and 2010 saw prices within a relatively tight range of \$3.00 to \$4.50
per bushel. In the final year of the sample, uncertainty and rapid price
increases reigned again as worries about a smaller than anticipated crop
yield and small ending stocks drove prices to nearly \$8.00 per bushel.
The number of transactions per day, depicted in the bottom panel,
appears to stay within a fairly stable band throughout the sample period
-- with perhaps an uptrend during the price spike of 2008 and a slight
upward trend toward the end of the sample.

The second and third panel display the number of revisions to the best
ask and best bid, respectively. The number of quote revisions is fairly
stable within a band of about 25,000 to 50,000 revisions from 2008 to
mid 2010. The exception being a brief period in late 2008 when the
market had bottomed after a dramatic fall from a high that summer of
nearly \$8.00 per bushel. Starting in the latter period of 2010, a
notable increase in the number of quote revisions, and the volatility of
the number of quote revisions can be observed. While they do not stay
within a well defined band, most days the number of quote revisions fall
within an range of 30,000 to 75,000. Because this does not appear to
coincide with a commensurate increase in the number of transactions
(depicted in the bottom panel of figure 1), one must assume this is due
to an increase in quoting strategies particularly suited to electronic
markets. A noticeable decrease in the number of transactions, and
especially the number of quote revisions is visible during the final
weeks of 2008, 2009, and 2010, corresponding with the Christmas and New
Year's holiday.

While price levels were volatile, the share of contracts traded on the
CME's electronic trading platform, Globex, had already stabilized to
nearly 90\% by 2008 (Peterson 2015). So any effects we study should not
be related to trading infrastructure changes that may have occurred
during the migration of volume to the electronic exchange. The data are
time-stamped to the second, but trades and updates to the top of the
book routinely occur more rapidly than once per second. This results in
several updates to the top of the book displaying the same time stamp.
This requires us to either aggregate to the second, or to simulate
sub-second time stamps (Hasbrouck 2015; X. Wang 2014). Since we
calculate correlations between updates to the top-of-the book for
several contract maturities, simulation would need to preserve (at
least) the order in which updates arrived to each respective contract.
Since preserving the order of arrival across different contracts is
impossible, we aggregate to the second.\footnote{We take the last entry
  on each time-stamp for the aggregation.}

Further, we exclude days on which there was a limit price move in any of
the contracts, since when prices are locked at the limit, correlations
are not meaningful (dates deleted due to limit price moves and the
corresponding information events, if known, are as follows: 1/12/2010,
revision to a Crop Production report; 3/31/2011, Prospective Plantings
report; 6/30/2011, Planted Acres report; 10/8/2010, World Agricultural
Supply and Demand Estimates (WASDE); and 12/9/2010, WASDE). Also, we
exclude 4/5/2010, because there was an unusually high number of
revisions to the best bid and best offer. Since we were not able to
process all of the data for this day in a reasonable amount of computing
time, we drop this day from our sample. Additionally, 7/5/2011 was an
unusually light trading day after the Fourth of July holiday and
resulted in no data for the third to mature contract, so we dropped this
day as well.

\section{Analysis}\label{analysis}

Our analysis considers the correlation of logged changes to quotes in
the nearby contract to logged changes to quotes in the deferred (1, 2,
and 3 maturities). We described, in the Conceptual Framework section,
that when information arrives to the market, it should affect the entire
forward curve in the same direction. In other words, information that
raises the best bid (offer) in the nearby contract, should raise the
best bid (offer) in the deferred contracts as well. Linkages between the
nearby and deferred contracts can be measured with simple correlations
in the absence of a formal model. One of the key methodological issues
when it comes to assessing high frequency data is the non-normality of
price data that complicates proper modeling and requires the use of
appropriate methodologies for analysis (A{ï}t-Sahalia, Mykland, and
Zhang 2005; Andersen et al. 2001; Easley, Prado, and O'Hara 2012;
Hasbrouck 2013; Lee and Mykland 2008; Lehecka, Wang, and Garcia 2014).

We have two primary objectives: 1) calculate the strength of
correlations between the order books of the nearby and deferred
contracts, and 2) measure (or bound) the time it takes for information
to be transmitted from nearby to deferred contracts. To measure the
first, we calculate contemporaneous (zero time lag) correlations between
the log changes of quotes in the nearby and the deferred contracts.
Then, to measure the second, we calculate the correlation between time
lagged log changes of quotes of the nearby with log changes of quotes of
the deferred contracts. We lag the nearby by one second and ten seconds.
The time lagged correlations provide a measure of how long it takes for
information to be transmitted from nearby to the deferred contracts. The
logic is that if we observe contemporaneous correlation between the
nearby and deferred contracts, we can search for the time lag at which
we observe the correlation disappearing. We conclude that information
has been fully transmitted when the time lagged nearby and deferred
contract order book revisions become uncorrelated. Conversely, we may
observe that there is no contemporaneous correlation, but there is
lagged correlation.

Since the corn futures contract experiences non-uniform trading volume
throughout the day, there may be time of day effects in the strength and
rate at which information is transmitted through the futures market. To
capture how the speed of information transmission changes throughout the
trading day, we divide the day into ten minute intervals starting at
9:30am Central Standard Time, the beginning of the daytime trading
session for CBOT corn futures. We calculate the correlations described
in detail below for each ten minute interval. Ten minutes was shown to
be long enough for market adjustment to take place in Lehecka (2014).
This allows us to detect if there are any discernible patterns to the
transmission of information over the trading day. Since one correlation
is calculated per day per ten minute interval, for every ten minute
interval we recover a distribution of correlations.

\subsection{Information-Based Trading Activity and Contemporaneous
Correlations in the Top of the
Book}\label{information-based-trading-activity-and-contemporaneous-correlations-in-the-top-of-the-book}

As mentioned, it is common to have multiple revisions to the order book
on the same second (and consequently receive the same time-stamp in the
data). The converse is also true, however. It is also common for a
number of seconds to transpire before the top of the order book is
revised - particularly in the middle of the daytime trading session.
This results in our variables containing many zeros. How these zeros are
distributed between the contracts is related to the concepts of
liquidity-based versus information-based activity discussed in the
conceptual framework.

To fix ideas, consider the possible outcomes when examining
contemporaneous log changes in the top of the book between the nearby
and the deferred contracts. There are three possibilities; on any time
stamp one of the three situations may occur: 1) neither the nearby nor
the deferred has a zero log change in the bid (offer), 2) either the
nearby or the deferred has a zero log change in the bid (offer), but not
both, or 3) both the nearby and the deferred have a zero log change in
the bid (offer).

Based on the definition of liquidity-based activity and
information-based activity in the conceptual framework, we present a
case for interpreting (1) as information-based activity, (2)
liquidity-based activity, and (3) liquidity-based activity.\footnote{In
  the third case, no activity at all is observed in the quoted price
  changes, but quoted quantities may have changed due to new limit
  orders arriving, limit orders being cancelled, or market orders
  arriving taking some of the quoted quantities off the book. This is
  indicative of liquidity-based activity as well.}

The intuition is that if both the nearby and deferred contracts
experience a revision in the same direction, they are likely responding
to the arrival of information to the marketplace, and best bids (offers)
adjust accordingly. This is in contrast to the case where one of the two
contracts experiences a revision to the bid (offer) and the other
contract has no change. If one contract experiences a revision in the
best bid (offer) and the other does not, it is likely that the revision
results from a liquidity-based order in traders' efforts to exit their
positions.

If this intuition is correct, it is informative to consider only
time-stamps for which both contracts experienced a revision - that is
isolating what we are referring to as information-based activity to case
(1) above.

\begin{equation} \label{corBB} 
corr^{Bid}_{tI} = \frac{\sum\limits_{i=1}^{n} \left(bid_{ti}^N - \overline{bid_t^N}\right) \left(bid_{ti}^D - \overline{bid_t^D}\right)}{\sqrt{\sum\limits_{i=1}^{n} \left(bid_{ti}^N - \overline{bid_t^N}\right)^2 \sum\limits_{i=1}^{n}\left(bid_{ti}^D - \overline{bid_t^D}\right)^2}} \: \textrm{such that} \: bid_{ti}^N \: \textrm{and} \: bid_{ti}^D \neq 0
\end{equation}

Equation \ref{corBB} indicates that we calculate the correlation between
the log change of the nearby best bids, \(bid_{ti}^N\), and the log
change of the deferred best bids, \(bid_{ti}^D\), for every day,
\emph{t}, and in every 10-minute interval in the daytime trading
session, \emph{I}, using the observations, \emph{i}, when both the
nearby and the deferred best bid log chnges are not equal to zero
(\(bid_{ti}^N \: \textrm{and} \: bid_{ti}^D \neq 0\)). The correlations
from equation (1) are calculated for the nearby and one deferred, nearby
and two deferred, and nearby and three deferred contracts.

Similarly equation \ref{corOO} indicates that we calculate the
correlation between the log change of the nearby best offer,
\(offer_{ti}^N\), and the log change of the deferred best offer,
\(offer_{ti}^D\) for every day, \emph{t}, and in every 10-minute
interval in the daytime trading session, \emph{I}, using the
observations, \emph{i}, when both the nearby and the deferred best offer
are not equal to zero,
\({offer_{ti}^N \: \textrm{and} \: offer_{ti}^D} \neq 0\).

\begin{equation} \label{corOO}
\begin{split}
& corr^{Offer}_{tI} = \frac{\sum\limits_{i=1}^{n} \left(offer_{ti}^N - \overline{offer_t^N}\right) \left(offer_{ti}^D - \overline{offer_t^D}\right)}{\sqrt{\sum\limits_{i=1}^{n} \left(offer_{ti}^N - \overline{offer_t^N}\right)^2 \sum\limits_{i=1}^{n}\left(offer_{ti}^D - \overline{offer_t^D}\right)^2}} \\
& \textrm{such that} \: {offer_{ti}^N \: \textrm{and} \: offer_{ti}^D} \neq 0 
\end{split}
\end{equation}

The correlations from equations \ref{corBB} and \ref{corOO} are
calculated for the nearby and one deferred, nearby and two deferred, and
nearby and three deferred contracts.

\subsection{Speed of Information Transmission and Time-Lagged
Correlations in the Top of the
Book}\label{speed-of-information-transmission-and-time-lagged-correlations-in-the-top-of-the-book}

To provide insights on the speed at which information is transmitted
from the nearby to the deferred contracts, we lag the nearby series of
log changes in the bid (offer) and calculate the correlation with the
deferred bids (offers). This allows us to determine the length of time
it takes for information to be fully transmitted to the deferred
contracts. The assumption is that the length of time it takes for the
revisions to the top of the nearby limit order book to become
uncorrelated with revisions to the top of the deferred limit order books
is the length of time it takes for information to be transmitted between
the two markets.

\begin{equation} \label{corLBB}
corr^{LagBid}_{tI} = \frac{\sum\limits_{i=2}^{n} \left(bid_{t(i-1)}^N - \overline{bid_t^N}\right) \left(bid_{ti}^D - \overline{bid_t^D}\right)}{\sqrt{\sum\limits_{i=1}^{n} \left(bid_{t(i-1)}^N - \overline{bid_t^N}\right)^2 \sum\limits_{i=1}^{n}\left(bid_{ti}^D - \overline{bid_t^D}\right)^2}} \: \textrm{such that} \: {bid_{t(i-1)}^N \: \textrm{and} \: bid_{ti}^D} \neq 0
\end{equation}

Equation \ref{corLBB} indicates that we calculate the correlation
between the lagged log change of the nearby best bid,
\(bid_{t(i-1)}^N\), and the log change of the deferred best bid,
\(bid_{ti}^D\) for every day, \emph{t}, and in every 10-minute interval
in the daytime trading session, \emph{I}, using the observations,
\emph{i}, when both the lagged nearby and the deferred best bid are not
equal to zero,
\({bid_{t(i-1)}^N \: \textrm{and} \: bid_{ti}^D} \neq 0\).

\begin{equation} \label{corLOO}
\begin{split}
& corr^{LagOffer}_{tI}  = \frac{\sum\limits_{i=1}^{n} \left(offer_{t(i-1)}^N - \overline{offer_t^N}\right) \left(offer_{ti}^D - \overline{offer_t^D}\right)}{\sqrt{\sum\limits_{i=1}^{n} \left(offer_{t(i-1)}^N - \overline{offer_t^N}\right)^2 \sum\limits_{i=1}^{n}\left(offer_{ti}^D - \overline{offer_t^D}\right)^2}} \\
& \textrm{such that} \: {offer_{t(i-1)}^N \: \textrm{and} \: offer_{ti}^D} \neq 0
\end{split}
\end{equation}

Similarly, equation \ref{corLOO} indicates that we calculate the
correlation between the lagged log change of the nearby best offer,
\(offer_{t(i-1)}^N\), and the log change of the deferred best offer,
\(offer_{ti}^D\) for every day, \emph{t}, and in every 10-minute
interval in the daytime trading session, \emph{I}, using the
observations, \emph{i}, when both the lagged nearby and the deferred
best offer are not equal to zero,
\({offer_{t(i-1)}^N \: \textrm{and} \: offer_{ti}^D} \neq 0\).

\subsection{Spread Trades, Information Transmission, and Time-Lagged
Bid-to-Offer (Offer-to-Bid)
Correlations}\label{spread-trades-information-transmission-and-time-lagged-bid-to-offer-offer-to-bid-correlations}

Surely the spread trade is an important component that keeps nearby and
deferred contracts linked in economically meaningful ways. However, a
spread trade is entered as a buy (sell) in the nearby and an sell (buy)
in the deferred contract. Until now, we have presented correlations
between bib-to-bid and offer-to-offer in the nearby and deferred
contracts. In equation \ref{corrLBO}, we define a measure of how
correlations between lagged log changes in the nearby bid and log
changes in the deferred offer measure the effect of spread traders in
transmitting information up the forward curve.

\begin{equation} \label{corrLBO}
corr^{LagBO}_{tI} = \frac{\sum\limits_{i=1}^{n} \left(bid_{t(i-1)}^N - \overline{bid_t^N}\right) \left(offer_{ti}^D - \overline{offer_t^D}\right)}{\sqrt{\sum\limits_{i=1}^{n} \left(bid_{t(i-1)}^N - \overline{bid_t^N}\right)^2 \sum\limits_{i=1}^{n}\left(offer_{ti}^D - \overline{offer_t^D}\right)^2}} \: \textrm{such that} \: {bid_{t(i-1)}^N \: \textrm{and} \: offer_{ti}^D} \neq 0
\end{equation}

Where we calculate the correlation between the lagged log change of the
nearby best bid, \(bid_{t(i-1)}^N\), and the log change of the deferred
best offer, \(offer_{ti}^D\), for every day, \emph{t}, and in every
10-minute interval in the daytime trading session, \emph{I}, using the
observations, \emph{i}, when both the lagged nearby and the deferred
best offer are not equal to zero, when both the lagged nearby offer and
the deferred best bid are not equal to zero,
\({bid_{t(i-1)}^N \: \textrm{and} \: offer_{ti}^D} \neq 0\).

\begin{equation} \label{corrLOB}
corr^{LagOB}_{tI} = \frac{\sum\limits_{i=1}^{n} \left(offer_{t(i-1)}^N - \overline{offer_t^N}\right) \left(bid_{ti}^D - \overline{bid_t^D}\right)}{\sqrt{\sum\limits_{i=1}^{n} \left(offer_{t(i-1)}^N - \overline{offer_t^N}\right)^2 \sum\limits_{i=1}^{n}\left(bid_{ti}^D - \overline{bid_t^D}\right)^2}} \: \textrm{such that} \: {offer_{t(i-1)}^N \: \textrm{and} \: bid_{ti}^D} \neq 0
\end{equation}

Similarly for equation \ref{corrLOB} we calculate the same correlations
as in equation \ref{corrLBO} except that we use the lagged nearby offer
and the deferred bid.

\subsection{USDA Announcement Days}\label{usda-announcement-days}

On USDA report announcement days there is often a significant amount of
information that market participants receive at the same time, causing
large price fluctuations and larger than usual trading volumes.
Therefore, in our analysis we also separate out days on which major USDA
reports are released and calculate the same correlations described
above. During our sample period, the USDA reports we include were
released at 8:30 am CST, before the day trading session began.

\section{Results}\label{results}

Table 2 contains a synopsis of the results that will be presented as
figures 2, 3, and 4. Figure 2 presents the strength of the correlation
between the nearby and deferred contracts by calculating the
contemporaneous correlation between log changes of nearby bids (offers)
and log changes of deferred bids (offers). Figure 3 presents the
strength of the correlation of log changes of nearby bids (offers) and
log changes of first deferred bids (offers) at time lags of 0, 1, and 10
seconds. Figure 4 presents the strength of the correlation of log
changes of nearby bids (offers) and log changes of first deferred offers
(bids) at time lags of 0, 1, and 10 seconds. Each figure is organized in
a similar way. The top pane shows correlations with the nearby bid, next
shows correlations with the nearby offer, next shows correlations with
the nearby bid on USDA report days, and finally the last pane shows
correlations with the nearby offer on report days. The dots represent
the mean of the distribution of calculated correlations and the bars
represent one standard deviation of the distribution of calculated
correlations.

\subsection{Information-Based Trading Activity and Contemporaneous
Correlations in the Top of the
Book}\label{information-based-trading-activity-and-contemporaneous-correlations-in-the-top-of-the-book-1}

In figure 2 contemporaneous correlation between the nearby and one, two,
and three deferred maturity contracts are displayed. Calculations are
made based on time-stamps where both the nearby and deferred maturity
experience non-zero revisions to the best bid (top panel) or offer
(second panel). The contemporaneous correlations between each nearby and
deferred contract pairs are very close to one for both best bids (top
panel) and best offers (second panel). The exception being that there is
a slight dip in correlations at the first and last ten minutes of the
trading day.

This implies that in the event both contracts experience revisions to
their respective limit order books, they are revised in lockstep. While
some of this correlation is artificially induced due to the tick
structure of price changes in this market (prices move in a minimum of
0.25 cent increments.), the correlations are too strong to attribute it
all to that. Additionally, since our data is only time-stamped to the
second, we may miss nuance that would be captured with data time stamped
to the millisecond. Regardless, the result is surprisingly strong and
indicates that information is largely transmitted up the forward curve
in less than one second. It is interesting that the distribution of
correlations between the nearby and 1, 2, and 3 deferred contract bids
are at such similar levels, hovering very close to one. Transmission of
information to the third deferred contract seems to be as strong as
transmission to the first deferred contract.

The bottom two panels of figure 2 are exactly analogous to the top two
except that they focus on USDA report days. We see a remarkably similar
depiction compared in that the correlations hover near one throughout
the trading day. If there had been a difference in the pattern of
correlations on USDA report days, one would expect the first ten minutes
of trading to display the largest effect. There is visibly more
variation in the means of these distributions, presumably as much a
result of the much smaller sample of report days versus non-report days.

We suspect two primary reasons that the full sample and USDA report day
results are so similar: 1) Since we removed days where the report
release corresponded to limit price moves, we systematically removed
report days where the most important information was conferred on the
market. It is possible that the remaining days corresponded to USDA
reports that were more easily translated to market impacts by traders,
and thus created results in figure 2 that look similar to a normal
trading day, while we systematically excluded report days containing
larger information shocks and presumably are harder for traders to
interpret the market impact of the report. 2) Since USDA reports were
released prior to the market open during this time period, the
information may have already been fully incorporated by market
participants by the time the market opened, resulting in no discernible
difference in the pattern of correlations in the first (and subsequent)
time bins.

\subsection{Speed of Information Transmission and Time-Lagged
Correlations in the Top of the
Book}\label{speed-of-information-transmission-and-time-lagged-correlations-in-the-top-of-the-book-1}

Figure 3 contains the correlations between log changes of the nearby and
log changes of one deferred contracts at 0, 1, and 10 second time lags,
when both experience non-zero changes. The graph shows the
contemporaneous correlation from figure 2 as a reference, and
correlations generated by lagging the nearby by one second and ten
seconds respectively. Here we expected to see a clear pattern of
decreased correlation as we increased the length of the time lag in the
nearby - reflecting that information is transmitted from nearby to
deferred contracts over a number of seconds. However, we see that the
correlation drops to zero with a lag of one second, which in this data
set is the shortest time lag possible.

There are three possible explanations for this. First, it is possible
that there is in fact a clear and decreasing correlation between lags of
the nearby and the deferreds, but it can only be observed on mili- or
micro-second time stamps. Then, when aggregating to the nearest second,
we observe contemporaneous correlation close to one, but zero
correlation even at the shortest possible time lag (one second).

Second, we explicitly assumed that price discovery happens in the nearby
contract when we lagged the nearby contract instead of the deferred
contract. If price discovery happens in the deferred contract, and takes
time to fully to be incorporated into the nearby contract, then we would
observe non-zero correlation between nearby quote revisions and lagged
\emph{deferred} quote revisions. When we did this, we observed a very
similar result as is presented in figure 3 - zero correlation at 1 and
10 second time lags of the deferred contract; this means there is not
evidence to support that price discovery happens in the deferred
contracts.\footnote{This figure is not presented in the interest of
  brevity.} Information seems to be fully transmitted within one second
to the first deferred contract.

Third, zero correlations between the deferred and time lagged nearby
would also occur if linkages between the nearby and deferred contracts
were immediately enforced by spread traders. This is examined in figure
4.

\subsection{Spread Trades, Information Transmission, and Time-Lagged
Bid-to-Offer (Offer-to-Bid)
Correlations}\label{spread-trades-information-transmission-and-time-lagged-bid-to-offer-offer-to-bid-correlations-1}

Figure 4 displays the means and error bars of the correlations between
log changes of the lagged nearby bid (offer) revisions and log changes
of the first deferred offer (bid) revisions. Here, as in figure 3 we see
contemporaneous correlations hovering near one. We observe that while
the correlations when the nearby is lagged by both one second and ten
seconds are near zero, the mean of the one second lagged correlations
are clearly higher than the ten second lags and larger than zero. It is
not compelling evidence, however, that we have observed incomplete
information transmission at the one second horizon. Though positive, it
is still quite close to zero.

\section{Conclusions}\label{conclusions}

Recent developments in commodity markets make it important to assess
price adjustment patterns with high frequency data. We focused this
paper on the corn market because it has experienced some of the most
pronounced changes in recent years. We gleaned insights from the
sequential trading market microstructure literature to generate metrics
of informed versus liquidity trading in commodity futures markets.
Sequential trading models allow liquidity providers to learn about the
existence of information arrivals and their directional implications for
security prices. From these models we infer that market makers detect no
new market information if we observe no changes to the best bid or best
offer in the limit order book. This is because in sequential trading
models, the market maker learns about the probability of an information
event from trader order flows and revises his breakeven bids and offers
accordingly.

We use simple correlations between non-zero log changes to the best bid
(offer) in the limit order book as our metric of information-based
activity in the market. Our results for CBOT corn indicate that the mean
contemporaneous correlation between non-zero changes to the nearby and
all deferred contracts was very close to 1 throughout the trading day.
When information arrives to the market, liquidity providers in contracts
of all maturities revise their bids and offers in lockstep (or in less
than one second) to reflect the new information.

To measure the speed of information transfer from nearby to deferred
maturities, we lagged the nearby by one and ten second respectively and
computed correlations in revisions to the best bid (offer) with deferred
contracts. We find that even at a one second lag, the shortest time lag
possible with this data set, the correlation between revisions to the
best bid and best offer dropped to zero.

These results indicate that we can learn much from trades and quotes
data about how many informed traders there are in a marketplace, but it
is clear that much additional work in this area is needed. We borrowed
from the sequential trading in securities literature, but future
research is needed to develop sequential trading models specific to
commodity futures markets. Such a model would provide richer insights
into trading and behavior of markets.

\section*{References}\label{references}
\addcontentsline{toc}{section}{References}

A{ï}t-Sahalia, Yacine, Per A Mykland, and Lan Zhang. 2005. ``How Often
to Sample a Continuous-Time Process in the Presence of Market
Microstructure Noise.'' \emph{Review of Financial Studies} 18 (2). Soc
Financial Studies: 351--416.

Akerlof, George A. 1970. ``The Market for` Lemons': Quality Uncertainty
and the Market Mechanism.'' \emph{The Quarterly Journal of Economics} 84
(3). JSTOR: 488--500.

Andersen, Torben G, Tim Bollerslev, Francis X Diebold, and Paul Labys.
2001. ``The Distribution of Realized Exchange Rate Volatility.''
\emph{Journal of the American Statistical Association} 96 (453). Taylor
\& Francis: 42--55.

Avalos, Fernando. 2014. ``Do Oil Prices Drive Food Prices? The Tale of a
Structural Break.'' \emph{Journal of International Money and Finance} 42
(0): 253--71.
doi:\href{http://dx.doi.org/http://dx.doi.org/10.1016/j.jimonfin.2013.08.014}{http://dx.doi.org/10.1016/j.jimonfin.2013.08.014}.

Bagehot, Walter. 1971. ``The Only Game in Town.'' \emph{Financial
Analysts Journal} 27 (2). CFA Institute: 12--14.

Balcombe, Kevin, A Prakash, and others. 2011. ``The Nature and
Determinants of Volatility in Agricultural Prices: An Empirical Study.''
\emph{Safeguarding Food Security in Volatile Global Markets}. Food;
Agriculture Organization of the United Nations (FAO), 89--110.

Boehmer, E., J. Grammig, and E. Theissen. 2007. ``Estimating the
Probability of Informed Trading-Does Trade Misclassification Matter?''
\emph{Journal of Financial Markets} 10 (1): 26--47.
doi:\href{http://dx.doi.org/10.1016/j.finmar.2006.07.002}{10.1016/j.finmar.2006.07.002}.

Brennan, Michael J. 1958. ``The Supply of Storage.'' \emph{The American
Economic Review} 48 (1). JSTOR: 50--72.

Cheng, Ing-Haw, and Wei Xiong. 2013. \emph{The Financialization of
Commodity Markets}. Working Paper 19642. Working Paper Series. National
Bureau of Economic Research.
doi:\href{http://dx.doi.org/10.3386/w19642}{10.3386/w19642}.

Copeland, Thomas E, and Dan Galai. 1983. ``Information Effects on the
Bid-Ask Spread.'' \emph{The Journal of Finance} 38 (5). Wiley Online
Library: 1457--69.

Easley, David, and Maureen O'Hara. 1987. ``Price, Trade Size, and
Information in Securities Markets.'' \emph{Journal of Financial
Economics} 19 (1). Elsevier: 69--90.

---------. 1992. ``Time and the Process of Security Price Adjustment.''
\emph{The Journal of Finance} 47 (2). Wiley Online Library: 577--605.

Easley, David, Soeren Hvidkjaer, and Maureen O'Hara. 2002. ``Is
Information Risk a Determinant of Asset Returns?'' \emph{The Journal of
Finance} 57 (5). Blackwell Publishing, Inc.: 2185--2221.

Easley, David, Nicholas M Kiefer, and Maureen O'Hara. 1997. ``One Day in
the Life of a Very Common Stock.'' \emph{Review of Financial Studies} 10
(3). Soc Financial Studies: 805--35.

Easley, David, Marcos M L{ó}pez de Prado, and Maureen O'Hara. 2012.
``The Volume Clock: Insights into the High-Frequency Paradigm (Digest
Summary).'' \emph{Journal of Portfolio Management} 39 (1). CFA
Institute: 19--29.

Gardebroek, Cornelis, and Manuel A. Hernandez. 2013. ``Do Energy Prices
Stimulate Food Price Volatility? Examining Volatility Transmission
Between \{US\} Oil, Ethanol and Corn Markets.'' \emph{Energy Economics}
40 (0): 119--29.
doi:\href{http://dx.doi.org/http://dx.doi.org/10.1016/j.eneco.2013.06.013}{http://dx.doi.org/10.1016/j.eneco.2013.06.013}.

Gilbert, Christopher L, and C Wyn Morgan. 2010. ``Food Price
Volatility.'' \emph{Philosophical Transactions of the Royal Society of
London B: Biological Sciences} 365 (1554). The Royal Society: 3023--34.

Glosten, Lawrence R, and Paul R Milgrom. 1985. ``Bid, Ask and
Transaction Prices in a Specialist Market with Heterogeneously Informed
Traders.'' \emph{Journal of Financial Economics} 14 (1). Elsevier:
71--100.

Hasbrouck, Joel. 2006. \emph{Empirical Market Microstructure: The
Institutions, Economics, and Econometrics of Securities Trading}. Oxford
University Press.

---------. 2013. ``High Frequency Quoting: Short-Term Volatility in Bids
and Offers.'' \emph{Available at SSRN 2237499}.

---------. 2015. \emph{High Frequency Quoting: Short-Term Volatility in
Bids and Offers}. Working Paper (available at Available at SSRN:
http://ssrn.com/abstract=2237499 or
http://dx.doi.org/10.2139/ssrn.2237499). New York University.
\href{ http://ssrn.com/abstract=2237499 }{http://ssrn.com/abstract=2237499}.

Henderson, Brian J., Neil D. Pearson, and Li Wang. 2015. ``New Evidence
on the Financialization of Commodity Markets.'' \emph{Review of
Financial Studies} 28 (5): 1285--1311.
doi:\href{http://dx.doi.org/10.1093/rfs/hhu091}{10.1093/rfs/hhu091}.

Irwin, Scott H., and Dwight R. Sanders. 2011. ``Index Funds,
Financialization, and Commodity Futures Markets.'' \emph{Applied
Economic Perspectives and Policy} 33 (1): 1--31.
doi:\href{http://dx.doi.org/10.1093/aepp/ppq032}{10.1093/aepp/ppq032}.

---------. 2012. ``Testing the Masters Hypothesis in Commodity Futures
Markets.'' \emph{Energy Economics} 34 (1): 256--69.
doi:\href{http://dx.doi.org/http://dx.doi.org/10.1016/j.eneco.2011.10.008}{http://dx.doi.org/10.1016/j.eneco.2011.10.008}.

Kumar, Praveen, and Duane J Seppi. 1994. ``Information and Index
Arbitrage.'' \emph{Journal of Business} 67 (4). JSTOR: 481--509.

Lee, Suzanne S, and Per A Mykland. 2008. ``Jumps in Financial Markets: A
New Nonparametric Test and Jump Dynamics.'' \emph{Review of Financial
Studies} 21 (6). Soc Financial Studies: 2535--63.

Lehecka, Georg V, Xiaoyang Wang, and Philip Garcia. 2014. ``Gone in Ten
Minutes: Intraday Evidence of Announcement Effects in the Electronic
Corn Futures Market.'' \emph{Applied Economic Perspectives and Policy}
36 (3). Oxford University Press: 504--26.

Mallory, Mindy L., Scott H. Irwin, and Dermot J. Hayes. 2012. ``How
Market Efficiency and the Theory of Storage Link Corn and Ethanol
Markets.'' \emph{Energy Economics} 34 (6): 2157--66.
doi:\href{http://dx.doi.org/http://dx.doi.org/10.1016/j.eneco.2012.03.011}{http://dx.doi.org/10.1016/j.eneco.2012.03.011}.

O'Hara, Maureen. 1995. \emph{Market Microstructure Theory}. Vol. 108.
Blackwell Cambridge, MA.

Peterson, Paul. 2015. ``How Will Closing the Trading Pits Affect Market
Performance.'' \emph{Farmdoc Daily} 5 (40).

Prakash, Adam, Christopher L Gilbert, and others. 2011. ``Rising
Vulnerability in the Global Food System: Beyond Market Fundamentals.''
\emph{Safeguarding Food Security in Volatile Global Markets}. Food;
Agriculture Organization of the United Nations (FAO), 45--66.

Serra, Teresa, and David Zilberman. 2013. ``Biofuel-Related Price
Transmission Literature: A Review.'' \emph{Energy Economics} 37 (0):
141--51.
doi:\href{http://dx.doi.org/http://dx.doi.org/10.1016/j.eneco.2013.02.014}{http://dx.doi.org/10.1016/j.eneco.2013.02.014}.

Trujillo-Barrera, Andres, Mindy Mallory, Philip Garcia, and others.
2012. ``Volatility Spillovers in US Crude Oil, Ethanol, and Corn Futures
Markets.'' \emph{Journal of Agricultural and Resource Economics} 37 (2):
247.

Vacha, Lukas, Karel Janda, Ladislav Kristoufek, and David Zilberman.
2013. ``Time--frequency Dynamics of Biofuel--fuel--food System.''
\emph{Energy Economics} 40 (0): 233--41.
doi:\href{http://dx.doi.org/http://dx.doi.org/10.1016/j.eneco.2013.06.015}{http://dx.doi.org/10.1016/j.eneco.2013.06.015}.

Vega, C. 2006. ``Stock Price Reaction to Public and Private
Information.'' \emph{Journal of Financial Economics} 82 (1): 103--33.
doi:\href{http://dx.doi.org/10.1016/j.jfineco.2005.07.011}{10.1016/j.jfineco.2005.07.011}.

Wang, Xiaoyang. 2014. ``Price volatility and liquidity cost in grain
futures markets.'' PhD thesis, University of Illinois.

Working, Holbrook. 1948. ``Theory of the Inverse Carrying Charge in
Futures Markets.'' \emph{Journal of Farm Economics} 30 (1). Oxford
University Press: 1--28.

---------. 1949. ``The Theory of Price of Storage.'' \emph{The American
Economic Review}. JSTOR, 1254--62.

Yan, Yuxing, and Shaojun Zhang. 2012. ``An Improved Estimation Method
and Empirical Properties of the Probability of Informed Trading.''
\emph{Journal of Banking \& Finance} 36 (2): 454--67.
doi:\href{http://dx.doi.org/http://dx.doi.org/10.1016/j.jbankfin.2011.08.003}{http://dx.doi.org/10.1016/j.jbankfin.2011.08.003}.

\end{document}


